\documentclass[12pt]{article}
% decent example of doing mathematics and proofs in LaTeX.
% An Incredible degree of information can be found at
% http://en.wikibooks.org/wiki/LaTeX/Mathematics


\addtolength{\oddsidemargin}{-.875in}
\addtolength{\evensidemargin}{-.875in}
\addtolength{\textwidth}{1.75in}

\addtolength{\topmargin}{-.875in}
\addtolength{\textheight}{1.75in}

\usepackage{amsmath}
\usepackage{upgreek}
\usepackage{amsthm,amssymb}
\usepackage{enumitem, mathtools}


%\renewcommand{\qedsymbol}{\rule{0.7em}{0.7em}}
\newcommand{\R}{\mathbb{R}}
\newcommand{\N}{\mathbb{N}}
\newcommand{\Q}{\mathbb{Q}}
\newcommand{\contradict}{\Rightarrow\!\Leftarrow}

\begin{document}
	\title{\vspace{-2.0cm}Midterm Study Guide Combinatorics}
	\maketitle
	
	\section{CH1}
	\section{CH2}
	\subsection{Unordered Selections 2.3}
	\textbf{THM 2.1.} $ \dbinom{n}{r} = \dbinom{n}{k-r} . \ ( 0 \leqslant r \leqslant n ). $ \\
	\textbf{THM 2.2.} Let n be a positive integer. Then, if $(1 + x)^n$ is expanded as a sum of powers of x, the coefficient of $x^r$ is $\dbinom{n}{r}$. \\
	ie. $(1 + x)^n = \dbinom{n}{0} + \dbinom{n}{1}x + \dbinom{n}{0}x^2 + ... + \dbinom{n}{n}x^n =  \sum\limits_{r=0}^{n}\dbinom{n}{r}x^r$. \\
	\textbf{THM 2.3.} $\dbinom{n}{r} = \dbinom{n-1}{r-1} + \dbinom{n-1}{r}$. \\
	\textbf{THM 2.4.} $(a+b)^n = \dbinom{n}{0}a^{n-1}b + \dbinom{n}{1}a^{n-2}b^2 + ... + \dbinom{n}{n}b^n = \sum\limits_{r=0}^{n}\dbinom{n}{r}a^{n-r}b^r$. \\ 
	\textbf{THM 2.5.} If n is any positive integer, then 
		\[ (1-x)^n  = 1 - \dbinom{n}{1}x + \dbinom{n}{2}x^2 - ... + (-1)^n\dbinom{n}{n}x^n = \sum\limits_{r=0}^n\dbinom{n}{r}(-1)^rx^r. \] \\
	\textbf{THM 2.6.} If n is any positive integer, then 
		\[ (1-x)^{-n}  = 1 + \dbinom{n}{1}x + \dbinom{n+1}{2}x^2 + \dbinom{n+2}{3}x^3 + ... = \sum\limits_{r=0}^{\infty}\dbinom{n+r-1}{r}x^r.\] \\ 
	\section{CH3}
	\subsection{Pairing within a set 3.1}
	\textbf{THM 3.1.} Let S be a set of mn objects. Then S can be split up (partitioned) into n sets of m elements in 
	\[ \frac{(mn)!}{(m!)^nn!} \]
	
	\section{CH4}
	\section{CH5}
	\section{CH6}
	\subsection{Block Designs 6.1}
	\textbf{DEF:} $ (b,v,r,k,\lambda)-design $ \\
	b - subsets (blocks),\\
	v - elements (varieties),\\
	r - each element is in exactly r blocks, \\
	k - each subset has k elements, \\
	$ \lambda $ - each pair of elements appears in $ \lambda $ subsets\\\\
	\textbf{THM 6.1:} In a block design each element lies in exactly r blocks, where
	\[ 	r(k-1) = \lambda(v-1) \]
	\textbf{THM 6.2:}  For a  $ (b,v,r,k,\lambda)-design $, 
	\[ b \geq v \]
	
	\subsection{Square Block Designs 6.2}
	\textbf{Properties} $ (v,k,\lambda)-design's $ incidence matrix has the following properties: 
	\begin{enumerate}
		\item Any row contains $ k $ 1's.
		\item Any column contains $ k $ 1's
		\item Any pair of columns both have 1's in exactly $ \lambda $ rows
		\item Any pair of rows both have 1's in exactly $ \lambda $ columns.\\
	\end{enumerate}
	\textbf{THM 6.3: } If A is a square (0, 1)-matrix (i.e. a matrix all of whose	entries are 0 or 1) and if A satisfies 
	\[
		A^TA = (k-\lambda)I - \lambda J
	\] 
	with $ k > \lambda$, then 
	\[
		AA^T = (k-\lambda)I - \lambda J
	\] 
	also holds. \\\\
	\textbf{DEF: } A finite projective plane of order $ q $ is defined to be a $ (v,k,\lambda)-configuration $ with the properties: 
	\begin{enumerate}
		\item $ v = q^2 + q  + 1 $
		\item $ k = q + 1 $
		\item $ \lambda = 1  $
	\end{enumerate}
	\textbf{fpp properties: } A fpp of order $ q $ has the following properties
	\begin{enumerate}
		\item Any line contains $ q + 1 $ points
		\item Any point lies on $  q+1 $ lines
		\item Any pair of points are joined on exactly one line
		\item Any pair of lines intersect in exactly one point
	\end{enumerate}
	\textbf{other knowledge about ffp's}
	\begin{enumerate}
		\item[a.] A plane of order $ q $ definitely exists if $ q\geq 2 $ is a prime or a power	of a prime.
		\item[b.] No plane of any other order is known to exist.
		\item[c.] There is definitely no plane of order 6, or in general of any order
		$ n $, where $ n $ is of the form $ (4k + 1) $ or $ (4k + 2) $, and is divisible an odd	number of times by a prime of the form $ (4h + 3) $.
	\end{enumerate}
	\textbf{FACT:} There is no finite projective plane of order 6.
	\subsection{Hadamard configurations}
	\textbf{DEF: } A $ (v,k,\lambda)-configuration $ is called a Hadamard configuration  when $ v =
	4m-1 $, $ k = 2m -1 $, $ \lambda =	m - 	1 $ for some integer $ m\geq2 $. \\
	\textbf{idea: } Hadamard Matrix is formed from taking the incidence matrix of a Hadamard configuration and changing the 0's to 1's.\\
	\textbf{DEF: } A  $ nxn $ matrix is a Hadamard matrix of order $ n $ if:
	\begin{enumerate}
		\item$  a_{ij} = \pm1, \forall i,j$
		\item $ AA^T = nI $
	\end{enumerate}
	\textbf{FACT: } Given a Hadamard matrix, it is permissible to interchange any two	rows or any two columns, or to multiply any row or column by -1, for	these operations do not effect the properties required by the definition.\\\\
	\textbf{THM 6.5: } If A is an $ n\times n $ Hadamard matrix with $ n>2 $, then
	$ n= 4m $ for some positive integer $ m $. Further, each row has exactly $ 2m $ +1s
	and $ 2m $ -1s, and, for any two chosen rows, there are exactly m columns in which both rows have +1.\\\\
	\textbf{THM 6.6: } Each normalized Hadamard matrix A of order $ 4m \geq 8 $
	yields a $ (4m -1, 2m -	1,m -	1)-configuration $. \\\\
	\textbf{Process: } A $ mn\times mn $ Hadamard matrix can be formed by taking 2 Hadamard matrices, $ A $ and $ B $, of order $ m $ and $ n $ respectively, and place $ A $ at every 1 in $ B $ and $ -A $ for every -1 in $ B $.
	\subsection{Error-correcting codes}
	\textbf{THM 6.7: } A code will detect all sets of $ h $ or fewer errors if any	two words differ in at least $ (h + 1) $ places.\\\\
	\textbf{THM 6.8: } A code will correct all sets of $ h $ or fewer errors if any two words differ in at least $ (2h + 1) $ places.
	\section{CH7}
	
\end{document}

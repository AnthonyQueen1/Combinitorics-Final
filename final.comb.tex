\documentclass[12pt]{article}
% decent example of doing mathematics and proofs in LaTeX.
% An Incredible degree of information can be found at
% http://en.wikibooks.org/wiki/LaTeX/Mathematics


\addtolength{\oddsidemargin}{-.875in}
\addtolength{\evensidemargin}{-.875in}
\addtolength{\textwidth}{1.75in}

\addtolength{\topmargin}{-.875in}
\addtolength{\textheight}{1.75in}

\usepackage{amsmath}
\usepackage{upgreek}
\usepackage{amsthm,amssymb}
\usepackage{enumitem, mathtools}


%\renewcommand{\qedsymbol}{\rule{0.7em}{0.7em}}
\newcommand{\R}{\mathbb{R}}
\newcommand{\N}{\mathbb{N}}
\newcommand{\Q}{\mathbb{Q}}
\newcommand{\contradict}{\Rightarrow\!\Leftarrow}

\begin{document}
	\title{\vspace{-2.0cm}Midterm Study Guide Real Analysis}
	\maketitle
	
	\section{ch1}
	\subsection{sup and inf 1.8}
	\bf DEF: Suppose $ S $ is an ordered set, $ E \subset S $, and $ E $ is bounded above. $ \alpha = \sup (S) $ has the following properties:
	\begin{enumerate}[label=(i.),leftmargin=.5in]
		\item $ \alpha $ is an upper bound of $ E $.
		\item if $ x < \alpha $ then $ x $ is not an upper bound of $ E $.
	\end{enumerate}
	\subsection{Least-upper-bound property 1.10}
	$ \mathbf{ DEF:} $ An ordered set $ S $ is said to have the least-upper-bound property if the following is true:
	$ E\subset S, E\neq\emptyset $, and $ E$ is bounded above, then $ sup (E )\in S $
	\subsection{THM 1.11}
	$ \mathbf{ THM:} $ Suppose S is an ordered set with the least-upper-bound property, $ B\subset S$, $ B $ is nonempty, and $ B $ is bounded below. Let $ L $ be the set of lower bounds of $ B $. Then 
	\[
	\sup(L) = \inf(B)
	\]
	That is to say that the supremum of the set of all lower bounds of B is equivalent to the infemum of B. 
	\subsection{thm 1.20}
	\begin{enumerate}[leftmargin=.5in]
		\item $ \mathbf {ARCHIMEDIAN PROPERTY:} $ If $ x\in\R $, and $ x>0 $, then there exists a positive integer n s.t.
		\[
		nx>y
		\]
		
		\item if $ x\in\R$, and $ x<y $, then there exists a $ p\in\Q $ s.t. $ x<p<y $.
	\end{enumerate}
	\subsection{Euclidean Spaces 1.36}
	$ \mathbf{ DEF:} $ 
\end{document}

\documentclass[12pt]{article}
% decent example of doing mathematics and proofs in LaTeX.
% An Incredible degree of information can be found at
% http://en.wikibooks.org/wiki/LaTeX/Mathematics


\addtolength{\oddsidemargin}{-.875in}
\addtolength{\evensidemargin}{-.875in}
\addtolength{\textwidth}{1.75in}

\addtolength{\topmargin}{-.875in}
\addtolength{\textheight}{1.75in}

\usepackage{amsmath}
\usepackage[thinlines]{easytable}
\usepackage{upgreek}
\usepackage{amsthm,amssymb}
\usepackage{enumitem, mathtools}


%\renewcommand{\qedsymbol}{\rule{0.7em}{0.7em}}
\newcommand{\R}{\mathbb{R}}
\newcommand{\N}{\mathbb{N}}
\newcommand{\Q}{\mathbb{Q}}
\newcommand{\contradict}{\Rightarrow\!\Leftarrow}

\begin{document}
	\title{\vspace{-2.0cm}Midterm Study Guide Combinatorics}
	\maketitle
	
	\section{CH1}
	\section{CH2}
	\subsection{Unordered Selections 2.3}
	\textbf{THM 2.1.} $ \dbinom{n}{r} = \dbinom{n}{k-r} . \ ( 0 \leqslant r \leqslant n ). $ \\
	\textbf{THM 2.2.} Let n be a positive integer. Then, if $(1 + x)^n$ is expanded as a sum of powers of x, the coefficient of $x^r$ is $\dbinom{n}{r}$. \\
	ie. $(1 + x)^n = \dbinom{n}{0} + \dbinom{n}{1}x + \dbinom{n}{0}x^2 + ... + \dbinom{n}{n}x^n =  \sum\limits_{r=0}^{n}\dbinom{n}{r}x^r$. \\
	\textbf{THM 2.3.} $\dbinom{n}{r} = \dbinom{n-1}{r-1} + \dbinom{n-1}{r}$. \\
	\textbf{THM 2.4.} $(a+b)^n = \dbinom{n}{0}a^{n-1}b + \dbinom{n}{1}a^{n-2}b^2 + ... + \dbinom{n}{n}b^n = \sum\limits_{r=0}^{n}\dbinom{n}{r}a^{n-r}b^r$. \\ 
	\textbf{THM 2.5.} If n is any positive integer, then 
		\[ (1-x)^n  = 1 - \dbinom{n}{1}x + \dbinom{n}{2}x^2 - ... + (-1)^n\dbinom{n}{n}x^n = \sum\limits_{r=0}^n\dbinom{n}{r}(-1)^rx^r. \] \\
	\textbf{THM 2.6.} If n is any positive integer, then 
		\[ (1-x)^{-n}  = 1 + \dbinom{n}{1}x + \dbinom{n+1}{2}x^2 + \dbinom{n+2}{3}x^3 + ... = \sum\limits_{r=0}^{\infty}\dbinom{n+r-1}{r}x^r.\] \\ 
	
	\begin{TAB}(r,1cm,2cm)[5pt]{|c|c|c|}{|c|c|c|}% (rows,min,max)[tabcolsep]{columns}{rows}
		Choose n from k & Number of ordered selections & Number of unordered selections \\ 
		Repetitions not allowed & $ \dfrac{n!}{(n-k)!} $ & $\dbinom{n}{k}$ \\
		Repetitions allowed & $n^k$ & $\dbinom{n+k-1}{k} $ \\
	\end{TAB}	
	
	\section{CH3}
	\subsection{Pairing within a set}
	\textbf{Properties} \\
	\text{(3.1)} The total number of different pairings of 2n objects is \[ \dfrac{(2n)!}{(2!)^nn!} \] 
	\text{(3.2)} If distinct representatives do exist, then, for every value of k; any k sets contain between them at least k elements \\
	\text{(3.3)}
	In general, a sequence $(a_1,a_2,...,a_n),a_1 \geqslant a_2 \geqslant ... \geqslant a_n$, of non-negative integers can be a score sequence only if  \\
	\text{(3.4)}
	\[ a_1 + a_2 + ... + a_n = \dbinom{n}{2} \] 
	and 
	\text{(3.5)} \[ a_{n-r+1} + a_{n-r+2} + ... + a_n \geqslant \dbinom{r}{2}   \]
	
	
	\textbf{THM 3.1.} Let S be a set of mn objects. Then S can be split up (partitioned) into n sets of m elements in 
	\[ \dfrac{(mn)!}{(m!)^nn!} \] 
	\textbf{THM 3.2.} If a graph has 2n verticies, each of degree $\geqslant$ n, then the graph has a perfect matching.
	\subsection{Pairings between sets}
	\textbf{THM 3.3.}(Philip Hall's theorem on distinct representatives). The sets $A_1,...,A_n$ possess a system of distinct representatives if and only if, for all k=$1,...,n$, any k $A_is$ contain at least k elements in their union.  \\
	\textbf{THM Assignment.} This assignment problem has a solution if and only if there is no value of k for which there are k jobs with fewer than k suitable applicants between them.		\\
	\textbf{THM Marriage.} Given a set of men and a set of women, each man makes a list of the women he is willing to marry. Then each man can be married off to a woman on his list if and only if, 
	\[ 
	(*)
	\begin{cases} 
	\text{for every value of k, any k lists contain in their union at least k names.}
	\end{cases}
	\]
	\textbf{THM 3.4.} If r $<$ n, any r x n Latin rectangle can be extended to an (r+1) x n Latin rectangle. \\
	\textbf{THM 3.5.}(Landau's theorem). the non-negative integers $a_1 \geqslant ... \geqslant a_n$ form the score sequence of a tournament if and only if conditions (3.4) and (3.5) are satisfied.  \\
	\textbf{THM 3.6.}(The Harem Theorem). Let $w_1,...,w_n$ be non-negative integers, and suppose that men $M_1,...,M_n$ each makes a list of the women he is willing to marry. Then each $M_i$ can be married to $w_i$ women on his list if and only if, for any subset $\{i_1,...,i_r\}$ of $\{ 1,...,n \}$, the lists of men $M_{i_1}, ..., M_{i_r}$ contain in their union at least $w_{i_1}, ..., w_{i_r}$ names. \\
	
	
	\section{CH4 Recurrence}
	\subsection{Misc. problems}
	\textbf{Q: }The problem of derangements. Suppose that n jobs have
	been assigned to n people. In how many ways can they be reassigned the	following day so that no person is given the same job as before?\\
	\textbf{A: }$ a_n = (n-1)a_{n-1} + (n-1)a_{n-2} $, and $ a_1 = 0 $, $ a_2 = 2 $, $ a_3 = 2 $\\
	\textbf{Explanation: } let $ n>2 $ and examine a derangement of $ 1, ..., n $. There are 2 Cases: 
	\begin{enumerate}
		\item $ n $ switches places with some other element $ r $, so there are $ (n-1) $ choices for $ r $, and $ a_{n-2} $ derangements of the remaining $ n-2 $ elements
		\item $ r $ moves to the $ n^{th} $ place, and $ n $ does not move to $ r $'s place. Relabel $ n $ by $ r $. Now $ n $ is fixed, with $ (n-1) $ elements to derange, which can be done in $ a_{n-1} $ ways. 
	\end{enumerate}
	\subsection{Fibonacci-type relations}
	\textbf{THM: } Suppose $ a_1 $ and $ a_2 $ are given, and that 
	\[
		a_n = Aa_{n-1} + Ba_{n-2} \quad (n\geq 3)
	\]
	Then : 
	\begin{enumerate}
		\item if the roots $ \alpha, \beta $ of the equation $ x^2 = Ax + B $ are distinct, then 
		\[ a_n = K_1\alpha^n + K_2\beta^n \]
		where $ K_1 $ and $ K_2 $ are determined uniquely by $ a_1 $ and $ a_2 $.
		\item if $ x^2 = Ax + B $ has a repeated root, then 
		\[ a_n = (K_1 + K_2n)\alpha^n \]
	\end{enumerate}
	\subsection{Using Generating Functions}
	\textbf{Example: } Partitions of an integer. i.e.
	\[
		5 = 4+ 1 =		3 + 2 =		3 + 1 + 1=2 + 2 + 1=
		2+ 1 + 1 + 1 =
		1 + 1 + 1 + 1 + 1.
	\]
	Note that the partitions are always in decreasing order. Let $ p(n) $ be the number of partitions of $ n $, so $ p(5)=7 $. Let $ f(x) $ be the generating function: 
	\[
		f(x) = p(1)x + p(2)x^2+ p(3)x^3+ p(4)x^4 +...
	\]
	Now consider the expression 
	\[
		(1-x)^{-1}(1-x^2)^{-2}(1-x^3)^{-3}... = (1 + x + x^2 + x^3 + ...)(1 + x^2 + x^4 + x^6 + ...)(1 + x^3 + x^6 + x^9 + ...)
	\]
	The coeficient of $ x^n  $ in this expression is equal to $ p(n) $
	\subsection{Misc. Methods}
	\textbf{Example: } Binomial recurrence relation: 
	\[
		\dbinom{n}{k} = \dbinom{n-1}{k} + \dbinom{n-1}{k-1}
	\] with 
	\[
		\dbinom{n}{0} = 1, \quad \dbinom{n}{n} = 1
	\]
	We will take the recurrence function: 
	\[
		f(n,k) = f(n-1, k) + f(n, k-1)
	\]
	with the conditions: 
	\[
		f(1,k) = 1, \quad f(n, 1) = n
	\]
	and modify it so that it fits the pattern defined by the binomial recurrence. Define $ g $ by 
	\[
		f(n,k) = g(n+k, k)
	\]
	Then we have 
	\[
		g(n+k, k) = g(n+k-1, k) + g(n + k-1, k-1)
	\]
	with the boundary conditions 
	\[
		g(k+1, k) = 1, g(n+1, 1) = n
	\]
	If we let $ m= n+k $, then 
	\[ g(m,k) = g(m-1, k) + g(m-1,k-1) \]
	however the boundary conditions are not correct. So we will try $ u=n+k-1 $. Then let $ h $ be defined as, 
	\[
		f(n,k) = h(n+k-1, k) = h(u,k)
	\]
	Thus
	\[
		h(u,k) = h(u-1, k) + h(u-1, k-1)
	\]
	with conditions 
	\[
		h(k,k) = 1, \quad h(n,1)= n
	\]
	Now it follows that $ h(u,k) = \dbinom{u}{k} $ and finally, 
	\[
		f(n,k) = \dbinom{n+k-1}{k}
	\]
	\section{CH5}
	\section{CH6}
	\subsection{Block Designs 6.1}
	\textbf{DEF:} $ (b,v,r,k,\lambda)-design $ \\
	b - subsets (blocks),\\
	v - elements (varieties),\\
	r - each element is in exactly r blocks, \\
	k - each subset has k elements, \\
	$ \lambda $ - each pair of elements appears in $ \lambda $ subsets\\\\
	\textbf{THM 6.1:} In a block design each element lies in exactly r blocks, where
	\[ 	r(k-1) = \lambda(v-1) \]
	\textbf{THM 6.2:}  For a  $ (b,v,r,k,\lambda)-design $, 
	\[ b \geq v \]
	
	\subsection{Square Block Designs 6.2}
	\textbf{Properties} $ (v,k,\lambda)-design's $ incidence matrix has the following properties: 
	\begin{enumerate}
		\item Any row contains $ k $ 1's.
		\item Any column contains $ k $ 1's
		\item Any pair of columns both have 1's in exactly $ \lambda $ rows
		\item Any pair of rows both have 1's in exactly $ \lambda $ columns.\\
	\end{enumerate}
	\textbf{THM 6.3: } If A is a square (0, 1)-matrix (i.e. a matrix all of whose	entries are 0 or 1) and if A satisfies 
	\[
		A^TA = (k-\lambda)I - \lambda J
	\] 
	with $ k > \lambda$, then 
	\[
		AA^T = (k-\lambda)I - \lambda J
	\] 
	also holds. \\\\
	\textbf{DEF: } A finite projective plane of order $ q $ is defined to be a $ (v,k,\lambda)-configuration $ with the properties: 
	\begin{enumerate}
		\item $ v = q^2 + q  + 1 $
		\item $ k = q + 1 $
		\item $ \lambda = 1  $
	\end{enumerate}
	\textbf{fpp properties: } A fpp of order $ q $ has the following properties
	\begin{enumerate}
		\item Any line contains $ q + 1 $ points
		\item Any point lies on $  q+1 $ lines
		\item Any pair of points are joined on exactly one line
		\item Any pair of lines intersect in exactly one point
	\end{enumerate}
	\textbf{other knowledge about ffp's}
	\begin{enumerate}
		\item[a.] A plane of order $ q $ definitely exists if $ q\geq 2 $ is a prime or a power	of a prime.
		\item[b.] No plane of any other order is known to exist.
		\item[c.] There is definitely no plane of order 6, or in general of any order
		$ n $, where $ n $ is of the form $ (4k + 1) $ or $ (4k + 2) $, and is divisible an odd	number of times by a prime of the form $ (4h + 3) $.
	\end{enumerate}
	\textbf{FACT:} There is no finite projective plane of order 6.
	\subsection{Hadamard configurations}
	\textbf{DEF: } A $ (v,k,\lambda)-configuration $ is called a Hadamard configuration  when $ v =
	4m-1 $, $ k = 2m -1 $, $ \lambda =	m - 	1 $ for some integer $ m\geq2 $. \\
	\textbf{idea: } Hadamard Matrix is formed from taking the incidence matrix of a Hadamard configuration and changing the 0's to 1's.\\
	\textbf{DEF: } A  $ nxn $ matrix is a Hadamard matrix of order $ n $ if:
	\begin{enumerate}
		\item$  a_{ij} = \pm1, \forall i,j$
		\item $ AA^T = nI $
	\end{enumerate}
	\textbf{FACT: } Given a Hadamard matrix, it is permissible to interchange any two	rows or any two columns, or to multiply any row or column by -1, for	these operations do not effect the properties required by the definition.\\\\
	\textbf{THM 6.5: } If A is an $ n\times n $ Hadamard matrix with $ n>2 $, then
	$ n= 4m $ for some positive integer $ m $. Further, each row has exactly $ 2m $ +1s
	and $ 2m $ -1s, and, for any two chosen rows, there are exactly m columns in which both rows have +1.\\\\
	\textbf{THM 6.6: } Each normalized Hadamard matrix A of order $ 4m \geq 8 $
	yields a $ (4m -1, 2m -	1,m -	1)-configuration $. \\\\
	\textbf{Process: } A $ mn\times mn $ Hadamard matrix can be formed by taking 2 Hadamard matrices, $ A $ and $ B $, of order $ m $ and $ n $ respectively, and place $ A $ at every 1 in $ B $ and $ -A $ for every -1 in $ B $.
	\subsection{Error-correcting codes}
	\textbf{THM 6.7: } A code will detect all sets of $ h $ or fewer errors if any	two words differ in at least $ (h + 1) $ places.\\\\
	\textbf{THM 6.8: } A code will correct all sets of $ h $ or fewer errors if any two words differ in at least $ (2h + 1) $ places.
	\section{CH7}
	
\end{document}
